\documentclass{leptc}
\begin{document}


\chap{双语彩色笔记模版}

作者:\href{mailto:alileptc@gmail.com}{LePtC}

项目主页:\url{https://github.com/LePtC/LeNote }

Last compiled on {\yyyymmdddate\today} at {\hhmmsstime} [UTC+8]


\chap{安装}

\ent[install TeX]{安装\TeX 系统}
Windows 系统可选择安装
\href{http://miktex.org/download}{MiKTeX}
然后选择自动安装缺失的包,或直接安装
\href{http://www.ctex.org/CTeXDownload }{CTeX Full}
或 \href{http://www.ctan.org/tex-archive/systems/texlive/Images/ }{TeXLive iso} ,
前两者是把 \file{leptc.cls} 放到
\code{CTeX/MiKTeX/tex/latex/} 目录下,
然后在 MiKTeX 的 Settings 里面点 Refresh FNDB 即可,
后者是在 \code{texlive/2014/texmf.cnf} 末尾加上
\\ \code{TEXMFLOCAL = $SELFAUTOPARENT/../texmf-local,E:/blabla/(anypath)},
\\然后把\file{leptc.cls} 放到
\code{(anypath)/tex/latex/misc} 这个路径中,
在命令行执行 \code{texhash} 即可

\ent[compiler]{编译器}
只有 latex+dvipdfmx 或 xelatex 编译出的 pdf 能正确复制,
前者请参考文件 \file{leptc.sty}

dvipdfmx 方案本狸已停止更新,推荐使用 \XeTeX 方案,
xelatex 的编译命令及常用选项:

\code{xelatex --quiet --synctex=1 -interaction=nonstopmode $(NAME_PART).tex}

xelatex 需要多编译几遍才能正确生成书签,
可在所有编译完成后加入对
\file{.aux,.out,.log}等文件的清理命令

\com{xelatex.exe 等编译器均在
\code{CTeX/MiKTeX/miktex/bin/}
或 \code{texlive/2014/bin/win32} 目录下,
如果命令行没有此命令,可在命令中输入 exe 的完整路径,
或手动将路径添加到系统的环境变量并重启}

\ent[editor]{编辑器}
\href{http://tex.stackexchange.com/questions/339/latex-editors-ides }{各种编辑器的比较},
熟悉哪个就用哪个好啦,
初学者可以就用安装\TeX 系统时带的 TeXworks

萌狸用的是 notepad++, synctex 需要借助一个 dde 插件
cl-2-dde-1.0.exe ,其它编辑器各有不同的设置方法

\ent[reader]{阅读器} 推荐使用
\href{http://blog.kowalczyk.info/software/sumatrapdf/download-free-pdf-viewer-cn.html }{SumatraPDF}
来查看 pdf,有
\href{http://xhmikosr.1f0.de/sumatrapdf/ }{64 位版本}
\com{非官方的}

支持 synctex,需在 \code{InverseSearchCmdLine} 里填入相应编辑器反向查找的命令

\ent[tex file]{tex 文档}
新建filename.tex,存为 UTF-8 无 BOM 格式,
开头为 \verb|\documentclass{leptc}|,
然后就可以 \\ \verb|\begin{document}| 闭着眼睛写啦,
喵 \tld

\com{待解决:文档名不能有空格否则不能识别,
不能有中文否则会报错}




\chap{章节}

\begin{tabular}{lcll}

	章节
	&\com{效果见右上方\eq{\nearrow} }
	&\verb|\chap{中文}|
	&\com{说明\eq{\downarrow} }\\


	&\ent[\B Superconducting \B{QU}antum \B Interference \B Device]{超导量子干涉器}

	&\verb|\ent[entry]{词条} |
	&居中用 \verb|\entc| 		\\


	&\eng[English translation]{注英文}
	&\verb|\eng[English]{正文} |
	& 用 \verb|\engr| 则英文标在右侧 		\\


	&\enl{标签}
	&\verb|\enl{标签} |
	& 用于\enl{例},\enl{定理},\enl{推论}等		\\

	inline公式
	&\eq{f(x,y)=\frac{\e^x}{y}}
	&\verb|\eq{\frac{\e^x}{y}}|
	&放弃用\verb|$$|,配对容易出错	\\

	display公式
	&\eqd{f(x,y)=\frac{\e^x}{y}}
	&\verb|\eqd{\frac{\e^x}{y}}|
	&修改公式模式只需加一个 \verb|d|即可	\\

	&\com{注释}
	&\verb|\com{注释}|
	&多行注释用\verb|\coms{注\\释}|	\\

	证明
	&\eq{\vec{v}=\prv{\od{}{t}(r \ve{r})=}\dot r\ve{r}+r\dot \theta\ve{\theta}\quad}
	&\verb|\prv{blabla=}|
	&灰色的优先级低于自动高亮 	\\

	笔记间的链接
	&\link{颜色}
	&\verb|\link[笔记名]{章节名}|
	&同一笔记内的链接笔记名可省略	\\

	贴图
	&\figin[0.05]{ali}
	&\verb|\fig[相对页面宽度]{图片名}|
	&内置\verb|\figin|多图并排\verb|\figgg|	\\

\end{tabular}

\chap{排版}

\figr{ali.jpg}{0.1}
{
图文混排 \to 图文混排 \to 图文混排 \to 图文混排 \to 图文混排 \to 图文混排 \to 图文混排 \to 图文混排 \to 图文混排 \to 图文混排 \to 图文混排 \to 图文混排 \to 图文混排 \to
}
\begin{tabular}{ll}
{\ttfamily
\begin{lstlisting}[language={[LaTeX]TeX}]
\figr{ali.jpg}{0.1}
{
图文混排 \to
...
(所有左排的内容)
}
\end{lstlisting}}

&\coms{记得在最后一个右括号之后还要有一个换行\\
待解决:图文混排环境内不支持 listing?}\\

\end{tabular}






\chap{颜色}

模版对以下情况做自动高亮:

\prv{更新:暂时取消橙色, 都用绿色, 单位换成蓝色, 章节由红色改为紫色}
\ \\
\begin{tabular}{lccl}

	函数名\sout{橙色}
	&\eq{\sin(x+y),\exp[x+y]}
	&\verb|\e^{x+y},\exp[x+y]|
	&自然对数 \eq{\e^x} 变橙色,命令为 \verb|\e| \\

	算符绿色
	&\eq{\dif x,\Dif x,\delta x,\Delta x,\nabla x}
	&\verb|\dif x,\delta x,\nabla x|
	&默认自动高亮,不高亮用 \verb|\olddelta| \\

	物理单位\sout{紫色}
	&\eq{\oC,6.67\E{-11}\uni{m^3/(kg\cdot s^2)}}
	&\verb|\uni{m^3/(kg\cdot s^2)}|
	&虚数单位 \eq{\ii} 变紫色,命令为 \verb|\ii| \\

\end{tabular}

\ \\
但字母作大型运算符\com{如\eqd{\min_{i=1}^n}}不做高亮,
不易混淆的符号型算符\com{如\eqd{\sqrt{\ }}}不做高亮


\chap{字体}

正文默认字体: Adobe 仿宋,\textbf{词条 Adobe 黑体},
英文 Times New Roman,\engr[Verdana]{英文翻译}

\prv{2015.05 更新:为改善斜杠的显示 \ent{例/例}, 黑体字体改为方正准圆, 不需要的请自行改回去}

打字机 \verb|\texttt{}|用于源代码: \file{file.tex}

\ \\
为了避免命名空间冲突,为了世界的和平,强迫症如下规定数学字体的含义:

\ \\
\begin{tabular}{lcl}

	所有变量、粒子符号为斜体
	&\eq{x,y,z,r,v,a,e,n,p}
	&\com{公式环境下默认为斜体} \\

	其它字母、元素符号为正体
	&\eq{\kB,\NA,F\inter,\cc,\mathrm{He}}
	&\verb|\mathrm{}| \\

	双线体注册为数域
	&\eq{\mathbb{N,Z,Q,A,R,C,H}}
	&\verb|\mathbb{}| \\

	花体注册为泛函 %和大O记号?
	&\eq{\mathcal{L,F,Z}}
	&\verb|\mathcal{}| \\

  粗体注册为群
  &\eq{\mathbf{U}(n),\mathbf{SU}(2),\mathbf{T}^\alpha}
  &\verb|\mathbf{}| \\

  哥特体注册为代数
  &\eq{\mathfrak{su}(n),\mathfrak{so}(2)}
  &\verb|\mathfrak{}| \\

\end{tabular}



\newpage
\chap{数学}

\begin{tabular}{lcl}

	矢量
	&\eq{\vec{OA},\vec{p_c}',\vecd{p},\ve{r}}
	&\verb|\vec{OA},\vec{p_c}',\vecd{p},\ve{r}|\\

	张量
	&\eq{\vvecd{T},\vvvec{\varepsilon}}
	&\verb|\vvecd{T},\vvvec{\varepsilon}|\\

	矢量算符
	&\eq{\hatv{p},\hatvs{S}}
	&\verb|\hatv{p},\hatvs{S}|\\

	导数,偏导数
	&\eqd{\od{y}{x},\pd[2]{L}{x},\md{L}{4}{x}{2}{y}{2}}
	&\verb|\od{y}{x},\pd[2]{L}{x},\md{L}{4}{x}{2}{y}{2}|\\

	某处的导数
	&\eq{\odat{y}{x}{x_0},}
	\eqd{\odat{y}{x}{x_0},\pdat{L}{x}{y,z}}
	&\verb|\odat{y}{x}{x_0},\pdat{L}{x}{y,z}|\\

	矢量微分
	&\eq{\nabla x,\nablad \vec x,\nablat \vec x,\nablas x}
	&\verb|\nabla x,\nablad \vec x,\nablat \vec x,\nablas x|\\

	矩阵,行列式
	&\eq{\mat{1&0\\0&1},\matd{-a&b\\c&-d}}
	&\verb|\mat{1&0\\0&1},\matd{-a&b\\c&-d}|\\

	左花括号
	&\eq{\delta _{ij} = \leftB[2]{\matn{1 &(i = j)\\ 0 &(i \ne j)}}}
	&\verb|\leftB[行数]{\matn{1 &(i = j)\\ 0 &(i \ne j)}}|\\

\end{tabular}

\ \\
太多了 ... 慢慢写



\chap{学习网站}

\url{http://tex.stackexchange.com/ }

\href{http://linux-wiki.cn/wiki/zh-hans/LaTeX%E4%B8%AD%E6%96%87%E6%8E%92%E7%89%88%EF%BC%88%E4%BD%BF%E7%94%A8XeTeX%EF%BC%89 }{ LaTeX中文排版(使用XeTeX)}

\href{http://www.wikibooks.org }{维基 book}






\end{document}
